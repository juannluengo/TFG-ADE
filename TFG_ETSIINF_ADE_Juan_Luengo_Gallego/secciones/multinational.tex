\chapter{Auditing in Multinational Corporations}
\label{ch:multinationals}

Multinational Corporations (MNCs), the main subject of this research, are, as explained by A. A. Lazarus \cite{MCsDefinition}, organizations in the business area whose activities can be attributed to multiple countries, hence the name, multiple nations. It can also be considered the organizational form that defines its direct foreign investment. In this \cref{ch:multinationals}, these MNCs are studied. With their diverse and often multicurrency operations, MNCs face unique challenges related to auditing.

\section{Characteristics of Multinational Corporations}

In current times, Multinational Corporations are seen everywhere with their presence constantly increasing and without intentions of slowing the trend. As explained by Joseph T. Martelli et al. \cite{growthMNCs}, MNCs have experimented with big changes in their compositions throughout the years and their returns have significantly increased thanks to globalization. In his study, the 500 biggest companies in the USA in 2009 were analyzed. In his own words, "The world’s largest companies listed on the 2009 Fortune listing are represented in five of the seven continents." Revealing that Europe holds the biggest share of MNCs with 36.2\%, followed by North America with 31.4\% and Asia with 29.2\%. In addition, the sector most represented, with 20.2\% of businesses, is the financial one, followed by the industries with 16.8\%, discretionary consumption (non-essential products) with 12.8\%, and the energy sector with 12.8\%. \\

Multiple factors lead to MNCs growth, like the international experience of its directors and the international network of connections as explained by Danchi Tan et al. \cite{DataGrowthMNCs}. A big part of it can be attributed to technological know-how (Kirca et al. \cite{technicalKnowHow}). This includes mainly R\&D investments, which lead to new technologies like more precise and cost-effective auditing software or a better logistics system. Furthermore, the bigger these companies get, the more resources they have to invest in R\&D, and the bigger they can get. In retrospect, is precisely that technological capacity that maintains MNCs maintain their competitiveness and profitability in the global markets. \\

Multinational Corporations are also responding to external conditions, with the improvement of infrastructure for example (T. Martelli et al. \cite{growthMNCs}). Moreover, this is not only beneficial for the MNCs, but also to the countries that host them. This term, baptized by Magnus Blomström et al. \cite{spilloverMNCs} as spillover refers to the advantages that emerging countries get for collaborating with MNCs. Between these perks, the local firms gain technological knowledge, international know-how and access to global financial markets. 

\subsection{Geographical and cultural complexity}

Being dispersed throughout the globe carries big opportunities, but also plenty of challenges to overcome. Klaus E. Meyer et al. \cite{multipleEmbedeness} explain this using the term "Multiple Embeddedness" which comes to describe the need of the MNCs to integrate in internal networks and in the local contexts of the host countries. In order to take advantage of the differences and likenesses of its multiple locations, MNCs have to manage a portfolio of different subsidiary activities in the local economy. The main challenge comes from balancing their subsidiaries' internal inlay, with the external one in the local country. This, often ends up, with a dilemma between their strategic goals and the local identity.\\

Another thing to take into consideration is the constantly changing institutional frameworks, and the usually scarce resources available in the region. Exchanges between companies are highly affected by different legal systems, politics, and bureaucracy. Jan Wouters et al. \cite{MNCsStrugglesInternationalLaw} describes this situation. Wouters also emphasized a few aspects that come from this ambiguity in different legal systems. Different non-binding agreements and instruments, like the OECD Guidelines for Multinational Enterprises \cite{OECDGuidelines} and the UN Guiding Principles on Business and Human Rights \cite{UNGuidingPrinciples}, have intentions to regulate MNCs conduct and acting. However, these lack enforceability and end up relying on voluntary compliance, which further complicates everything.
But as the saying goes, all that glitters is not gold. This cooperation between MNCs and (usually) developing countries creates a dependency situation in which the state may be reluctant to enforce certain regulations in order to keep the company's operations, and hence their jobs and economic activity, running. 


\section{Current Auditing Processes}
\label{Types of Auditing}

There are different types of auditing processes depending on the business objectives. The biggest distinction is between internal and external auditing. The main characteristics, as explained by Fatmir Mehmeti \cite{ExternalInternalAuditing1} and Lois Munro et al. \cite{ExternalInternalAuditing2} are:
\begin{itemize}
    \item \textbf{Internal auditing:} is mainly used as an activity guarantee and independent consulting. The main objective is to be able to provide the company with enough data about its internal controls in order to make well-formed decisions (Antonio Jiménez-Blanco \cite{claseAuditoríaT3}). It offers a systemic and disciplined scope to evaluate and improve the efficiency and efficacy of the risk management, control, and governance processes. The auditors in charge of these reports work for the company. The main goals this type of auditing pursues are:
    \begin{itemize}
        \item Evaluate the risk management and internal controls.
        \item Ensure compliance.
        \item Improve efficiency.
        \item Prevent fraud.
    \end{itemize}
    \item \textbf{External auditing:} on the other hand, this type of auditing focuses more on providing an accurate and transparent view of a company's financial statements to third parties. With the intention of giving the external public enough verified and truthful data so they can form their own decisions regarding the company's future (Antonio Jiménez-Blanco \cite{claseAuditoríaT3}). The external auditors are not directly employed by the company and are usually hired by the stakeholders or the board of directors. All these measures are in place so they do not feel pressure to provide an impartial opinion. The main goals this type of auditing pursues are:
    \begin{itemize}
        \item Independence.
        \item Verification of the financial statements.
        \item Compliance with standards and regulations.
        \item Reporting findings.
    \end{itemize}
\end{itemize}

\subsection{Role of the principal auditor}

The role of the principal auditor, as described by the NSW Government Department of Education \cite{PrincipalAuditorNSW} is pivotal in maintaining the integrity and efficiency of a company's operations. 

The main responsibility of the principal auditor is to lead and conduct the auditing operations and reviews of the system. They are in charge of conducting different types of auditing such as legal, accounting, technological, etc. They advise on strategies and risk management control to prevent fraud and errors and promote compliance processes. 
In addition, they represent the Auditing Board in meetings and help solve problems related to business integrity. It is the one in charge of reviewing, evaluating, and, at the end, informing, about the system and management processes to guarantee compliance. Involves a high degree of ethical practice, as well as judgment and communication skills. 


XXXX Añadir algo aquí 



\subsection{Limitations of auditors}

As explained before in \cref{Types of Auditing}, auditing can be divided between internal and external. However, when looking more into its limitations, it makes a big difference whether it is one type or the other.

In the internal auditing of multinational corporations, the main limitations come from the scale and complexity of their operations. Another significant one has to do with the possible lack of independence and objectiveness. As KH Spencer Pickett \cite{definitionInternalAuditing} puts it, "Internal auditing is a profession which has always prided itself on being a service to management." Since internal auditors are employees from the same organization they are auditing, they may be subjected to unconscious biases or pressures that affect their impartiality. In addition, being a multinational company, the internal auditors need to know not only their area of operation but the whole company. And not just on a technical level, but being geographically dispersed brings other challenges such as cultural or linguistic barriers. 

On the other hand, the main challenges and limitations of external auditing are diverse and complex. The high costs associated with auditing in multinational companies tend to be prohibitive or difficult to manage for some companies. In addition, coordination challenges arise given the necessity to align auditing challenges in multiple jurisdictions with different regulatory norms. 
They are also subject to strict deadlines and obstacles to accessing complete information, which further complicates the process. 


\section{Operating and Financial Risks of Auditing in Multinationals}

When companies get bigger and start operating in multiple countries, a few challenges arise. On the operating level, the differences in regulations and accounting practices between countries make it more difficult to collect and verify information, mainly financial ones. Geographically, things do not get easier, as being dispersed in multiple countries further complicates the coordination, and supervision when needed, of the operations and, mainly, auditing traces. 


\subsection{Agency costs and monitoring}

Financially speaking, the change rates for different currencies imply one extra step when auditing. "Foreign exchange volatility challenges auditors for not only demanding more audit efforts but also imposing heightened audit risk" Yuyuan Chang et al. \cite{MultipleCurrencyRisksAuditing}. This risk, as Yuyuan Chang et al. explained, comes from two paths: multiple currency operations increasing cash flow uncertainty and reduction of the reporting quality. In other words, foreign exchange risk, leads to increased audit efforts, hence augmenting fees.

The short explanation for this is that financial statements have to reflect transactions in the reporting currency, while, at the same time, exchange rate fluctuations can make significant variances in financial reports. But, exchange rates, are not always something to run away from. They are vital for a country's economy, as they affect the number of imports and exports. 

There are currently two ways to offset this risk. The first one, as explained by Yuyuan Chang et al. \cite{MultipleCurrencyRisksAuditing}, comes from the auditor experience and expertise in the international sector. The second one, more technical, is by using a financial instrument called financial hedging. As explained by LME \cite{financialHedgingDefinition}, the goal of financial hedging is to offset potential losses (in this case, derived from foreign exchange rates) by making another investment.


