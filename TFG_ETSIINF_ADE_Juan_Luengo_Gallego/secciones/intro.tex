\chapter{Introduction}
\label{ch:intro}

In an increasingly globalized and competitive business world, multinational corporations face a significant challenge in the management and supervision of their operations, mainly when finance-related. Traditional auditing processes, pivotal to guarantee the integrity and transparency of financial information, have limitations that affect their efficiencies and efficacy. These methods are not able to analyze the total sum of transactions of big corporations and highly base their operations on the internal controls of the company. As a result, the auditors, use sampling techniques that vary depending on the quality of the company's internal controls as explained by Wally Smieliauskas \cite{sampling}. This, as expected, carries long processes and an unfinished review of the practical majority of transactions. In addition, the scalability of these processes is directly related to the workforce in place. \\

Integrating new technologies in the auditing process has been a constant challenge. Current technology is not prepared to incorporate emerging tools such as Blockchain or Artificial Intelligence (AI) in an efficient way. This hardness the way to maximalize the potential to improve the precision and speed of auditing. Additionally, as explained by Maria Krambia-Kapardis \cite{fraudAuditing}, auditors "have legal responsibilities for detecting and reporting fraud". In traditional models, detection mechanisms are difficult to implement and scale, as not all transactions are going to be analyzed and the information is opaque for most of the organization. \\ 

In this context, blockchain technology aspires to be the solution (Derrick Bonyuet \cite{blockchainCharacteristics}). Its decentralized nature, immutability, and transparency offer a unique opportunity to transform the auditing processes by allowing a real-time analysis of all transactions. However, not all companies would allow these characteristics, as some may adapt the technology to their needs by making it cloud-based (centralized), more private, and without anonymity. But in that case, would it still be blockchain? \\

The objective of this thesis is to explore the implementation of blockchain technology in multinational corporations for real-time auditing. The specific objectives of this work will be:
\begin{itemize}
    \item Understand the context of current auditing practices and methods.
    \item Analyse the main challenges, with a focus on regulatory issues, technological barriers, and resistance from the people.
    \item Evaluate opportunities for blockchain technology in enhancing auditing processes.
    \item Describe historical data regarding the automation of auditing processes to critique real implementations and highlight lessons learned.
    \item Analyse the future trends and viability of multinational corporations.
\end{itemize}

The structure of the document is the following: Firstly, introspection about the current state of auditing procedures in multinational corporations will be done in \cref{ch:multinationals}. An explanation of the blockchain and its use in the business sector will follow in \cref{ch:blockchain}, including the different available architectures and the consensus mechanisms. After that, a deeper analysis of ledger technology and transaction processing will be made in \cref{ch:ledger}. Finally, some results and conclusions will be stated and their impact evaluated in \cref{conclusion} and \cref{ch: impact}. 

